\subparagraph*{Projet de parsing de recette au format json.}

\href{https://nodesource.com/products/nsolid}{\tt }

\href{https://etulab.univ-amu.fr/f19003179/projet-recette-json}{\tt }\hypertarget{autotoc_md0_autotoc_md1}{}\section{Présentation du projet}\label{autotoc_md0_autotoc_md1}
Cette application s\textquotesingle{}inscrit dans le cadre du projet d\textquotesingle{}Interface Homme-\/\+Machine. Ce projet à été commandité par Mr. Raffin dans le cadre du module \href{https://fr.wikipedia.org/wiki/Interactions_homme-machine}{\tt I\+HM}. Ce projet à été réalisé par deux étudiants en D\+UT Informatique sur le site \href{https://fr.wikipedia.org/wiki/Arles}{\tt d\textquotesingle{}Arles}.\hypertarget{autotoc_md0_autotoc_md2}{}\subsection{Fonctionnalités}\label{autotoc_md0_autotoc_md2}

\begin{DoxyItemize}
\item Importer un fichier au format \href{https://fr.wikipedia.org/wiki/Portable_Document_Format}{\tt json}
\item Changer l\textquotesingle{}étape de la recette en appuyant sur les boutons suivant/précédent
\item Calcul interne du temps nécessaire à la préparation de la recette
\item Affichage des ingrédients nécessaires à la préparation de la recette
\item Informations et rapport avec la recette et bien d\textquotesingle{}autres encore...
\end{DoxyItemize}\hypertarget{autotoc_md0_autotoc_md3}{}\subsection{Fonctionnement et Classes}\label{autotoc_md0_autotoc_md3}
Pour plus de précision sur le fonctionnement interne de l\textquotesingle{}application et les interractions entre les classes veuillez consulter la documentation \href{https://fr.wikipedia.org/wiki/Portable_Document_Format}{\tt pdf}.

Développé en mai 2020 par Lucas Pollet et Arsène Fougerouse 