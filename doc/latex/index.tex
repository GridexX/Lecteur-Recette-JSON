Cette application s\textquotesingle{}inscrit dans le cadre du projet d\textquotesingle{}Interface Homme-\/\+Machine. Ce projet à été commandité par Mr. Raffin dans le cadre du semestre 2 de D\+UT Informatique sur le site d\textquotesingle{}Arles.\hypertarget{index_utilite}{}\section{A quoi sert ce programme ?}\label{index_utilite}
Au lancement de l\textquotesingle{}application, l\textquotesingle{}utilisateur est amené à choisir un fichier de recette au format json qu\textquotesingle{}il viendra déposer / sélectionner via le pushbouton en bas.~\newline
 L\textquotesingle{}application ouvre ensuite une autre fenêtre affichant les informations relatives afin de réaliser la recette. ~\newline
 {\bfseries Prérequis \+:} Le programme nécessite d\textquotesingle{}être lancé sur un ordinateur {\itshape Windows} ou {\itshape Linux} et d\textquotesingle{}avoir un compilateur C++/ Qt Creator d\textquotesingle{}installé. Voir la doc \href{http://www.mingw.org}{\tt Min\+GW} pour plus d\textquotesingle{}informations.\hypertarget{index_mentions}{}\section{Mentions légales}\label{index_mentions}
Copyright(\+C) 2020-\/2021 \href{ mailto: lucas.pollet@etu.univ-amu.fr}{\tt Lucas Pollet} et \href{ mailto: arsene.fougerouse@etu.univ-amu.fr}{\tt Arsène Fougerouse}\hypertarget{index_references}{}\subsection{Référence des classes}\label{index_references}
Les classes sont référencées \href{file:///home/gridexx/projet-recette-json/doc/html/annotated.html}{\tt ici}. Pour plus de précision, veuillez voir la liste déroulante de l\textquotesingle{}onglet classe ou la documentation pdf. La conception a été spécialement pensé dans le but de la M\+VC. 